\newpage
\begin{center}
    \textbf{\large 1 ПРЕДМЕТНАЯ ОБЛАСТЬ И ТЕХНОЛОГИИ}
\end{center}
\addcontentsline{toc}{chapter}{1 ПРЕДМЕТНАЯ ОБЛАСТЬ И ТЕХНОЛОГИИ}

\textbf{1.1 Анализ предметной и проблемной области}

Предметной областью проекта является помощь абитуриентам в выборе подходящей профессии. 
Проект включает создание веб-приложения «Карьерные карты» для Московского политехнического университета, которое позволит 
абитуриентам оценивать карьерные перспективы различных профессий. Сайт предоставит информацию о востребованности профессий, 
возможных карьерных траекториях и вариантах трудоустройства для выпускников

В настоящее время абитуриентам Московского Политеха сложно определиться с выбором профессии из-за недостатка актуальной и 
структурированной информации о карьерных перспективах. Отсутствие удобного инструмента для анализа профессий усложняет процесс 
принятия решений, что может привести к неправильному выбору профессии и снижению мотивации к обучению.
Данное исследование будет посвящено анализу существующих информационных ресурсов и методов карьерного ориентирования, а также 
разработке рекомендаций по созданию веб-приложения, которое поможет абитуриентам оценить возможности карьерного роста и выбрать 
подходящую специальность на основе анализа рынка труда.

Информация играет ключевую роль в процессе выбора профессии. 
Доступность и качество данных о специальностях напрямую влияют на осознанность выбора абитуриентов. 
Исследования показывают, что персонализированные рекомендации увеличивают вероятность осознанного выбора профессии на 25\%. 
Это подчеркивает необходимость предоставления не только общей, но и адаптированной под индивидуальные запросы информации. 
Процедура тестирования, как отмечается, «автоматизирована и позволяет охватить большой контингент абитуриентов. 
Для них она необычна, интересна, особых сложностей не вызывает» (Источник, 7 с.). 
Такие подходы способствуют более глубокому пониманию различных профессий и поддерживают информированный выбор.

Индивидуализированный подход к предоставлению информации о профессиях может значительно повысить осознанность выбора 
абитуриентов. Согласно опросу 2022 года, 68\% студентов отметили, что доступ к информации о специальностях помог бы им 
сделать более осознанный выбор. Это подчеркивает необходимость разработки инструментов, учитывающих личные интересы и 
способности пользователей. Вместе с тем, «на сегодняшний день в научно-теоретической литературе и в художественной практике 
накоплен богатейший объем информационных ресурсов, однако до сих пор не вполне четко определен процесс работы над иллюстрацией 
литературного произведения, который будет способствовать развитию профессиональных навыков студентов как художников-педагогов» 
(Виданова, Гаврилов, 2016. 3 с.). Таким образом, важно интегрировать существующие ресурсы в индивидуализированные подходы для 
обеспечения более глубокого понимания профессий и их требований.

Цифровые технологии значительно способствуют повышению доступности информации о профессиях. По данным ЮНЕСКО, применение этих 
технологий в образовании увеличивает доступность информации для молодежи на 45\%. Веб-приложение, которое предоставляет 
структурированную и персонализированную информацию, может стать незаменимым инструментом для абитуриентов, позволяя им 
принимать более обоснованные решения. Вместе с тем «сегодня в системе довузовского образования приоритетными становятся 
развивающие технологии обучения, рассчитанные на постоянное самосовершенствование личности не только обучающегося, но и педагога» 
(Пахомова, б. г. 1 с.). Это подчеркивает значимость интеграции цифровых решений для формирования более информированного и 
подготовленного поколения.\\

\textbf{1.2 Анализ процесса выбора профессии}
