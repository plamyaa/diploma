\newpage
\begin{center}
  \textbf{\large АННОТАЦИЯ}
\end{center}

Наименование работы: веб-приложение для выбора профессии и построения карьерного пути абитуриента.

Цель работы: целью создания системы является снижение вероятности неправильного выбора специальности 
за счет повышения информированности абитуриентов о карьерных возможностях и перспективах. Также проект 
поможет студентам лучше понять требования к профессиям, необходимые навыки и карьерные перспективы.

Объект исследования: веб-приложение, помогающее абитуриентам выбрать профессию и построить карьерныйпуть 
на основе их интересов и карьерных перспектив.

Предмет исследования: разработка веб-приложения для помощи абитуриентам в выборе профессии и построении 
карьерного пути.

Работа состоит из Введения, трех глав, Заключения, Списка использованных источников и Приложений. Общий 
объем работы составляет X страниц, включая Y страниц Приложений. В работе содержится A рисунков, B 
таблиц и Z листингов кода. Библиография включает N источников.

Во Введении изложены цель, задачи, объект, предмет исследования, актуальность, новизна и практическая 
значимость работы. Первая глава посвящена анализу предметной области, выявлению основных проблем, 
изучению существующих аналогов и определению требований к веб-приложению. Вторая глава описывает 
проектирование системы, включая архитектуру веб-приложения, разработку схем базы данных и 
пользовательского интерфейса, а также описание разработки клиентской и серверной частей веб-приложения, 
реализацию алгоритмов генерации заданий и функционала для их выполнения студентами. Третья глава 
описывает процесс тестирования системы, обеспечение информационной безопасности, оценку удобства 
использования для преподавателей и студентов, а также стратегию продвижения. В Заключении представлены 
выводы по выполненной работе и перспективы дальнейшего развития системы.

\onehalfspacing
\setcounter{page}{6}

\newpage
\renewcommand{\contentsname}{\centerline{\large СОДЕРЖАНИЕ}}
\tableofcontents

\newpage
\begin{center}
  \textbf{\large ВВЕДЕНИЕ}
\end{center}
\addcontentsline{toc}{chapter}{ВВЕДЕНИЕ}


\textbf{Актуальность}

В современном мире выбор профессии является одним из ключевых этапов в жизни каждого человека. Этот 
процесс требует тщательной подготовки и анализа, так как от него зависит дальнейшее развитие и 
удовлетворённость жизнью. Однако многие абитуриенты сталкиваются с трудностями при принятии решения, 
что связано с недостатком информации о профессиях и карьерных перспективах. В условиях стремительного 
развития технологий и изменения рынка труда необходимость в инструментах, помогающих молодым людям принимать 
осознанные решения, становится особенно актуальной.

Целью данной работы является разработка веб-приложения, которое предоставит абитуриентам возможность 
получить доступ к информации о различных профессиях, их особенностях и требованиях. Задачи исследования 
включают анализ текущих проблем, связанных с выбором профессии, изучение существующих решений и определение 
функциональных требований к разрабатываемому приложению. Эти аспекты помогут создать инструмент, способный 
эффективно поддерживать абитуриентов в процессе принятия решения.

Цель данной работы заключается в создании веб-приложения, направленного на поддержку профессиональной 
ориентации. Для достижения поставленной цели необходимо решить следующие задачи:
\begin{enumerate}
\item Провести обзор аналогов веб-приложения;
\item Провести анализ процесса выбора профессии;
\item Провести анализ целевой аудитории веб-приложения;
\item Разработать требования к веб-приложению;
\item Спроектировать функционал веб-приложения;
\item Разработать модели данных веб-приложения;
\item Спроектировать пользовательский интерфейс веб-приложения;
\item Разработать клиентскую и серверную части веб-приложения;
\item Провести тестирование разработанного веб-приложения;
\item Предложить стратегию внедрения веб-приложения в образовательный процесс;
\item Описать перспективы развития веб-приложения.
\end{enumerate}

Объект исследования – веб-приложение, предназначенное для помощи абитуриентам в выборе профессии и построении 
карьерного пути.

Предмет исследования – процесс информирования абитуриентов о карьерных возможностях и перспективных направлениях 
обучения.

Таким образом, разработка данного веб-приложения позволит абитуриентам получить структурированную информацию о 
профессиях и карьерных возможностях, облегчая процесс выбора специальности. Внедрение системы повысит 
осведомленность пользователей о требованиях к разным профессиям и перспективных направлениях обучения, что 
поможет им принять более обоснованное решение о своем будущем.
